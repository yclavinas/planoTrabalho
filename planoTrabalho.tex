\documentclass[a4paper,10pt]{article}
\usepackage[utf8]{inputenc}
\usepackage{gensymb}
\usepackage[a4paper,bindingoffset=0.2in,%
left=1in,right=1in,top=1in,bottom=1in,%
footskip=.25in]{geometry}
\usepackage{blindtext}

%\def\floatpagefraction{0.95}
%\linespread{0.97}


\begin{document}
	
%	\maketitle
	
\section{Introdução}\label{introducao}


Esta pesquisa tem como objetivo melhorar a qualidade das análises realizadas
pelos sistemas computacionais para o estudo de imagens biomédicas de alta
resolução. O foco principal é direcionado para os sistemas que trabalham com a
análise de imagens de secções microscópicas de tecidos. Analisar essas imagens
possibilita o estudo da doença em nível celular ou sub-celular. Estas análises
podem auxiliar na caracterização da morfologia dos tecidos sub-celulares e
podem facilitar o entendimento dos mecanismos de doenças e melhorar a avaliação
da resposta aos tratamentos de uma doença. Isto é possível, pois muitas
informações sobre células e morfologia celular podem ser obtidas a partir das
análises destas imagens.

Entretanto, capturar e analisar essas informações em imagens de larga escala
mostra-se um desafio, uma vez que é necessário a revisão humana
subjetiva~\cite{kong2011comprehensive}. Como alternativa, a análise de imagens
computadorizadas provém uma oportunidade de observação das características
anatômicas de entidades biológicas em secções de imagens de
tecido~\cite{kong2010texture}.

Uma plataforma disponível para realizar análise de imagens médicas é a
desenvolvida pelo grupo do prof George Teodoro denominada \textit{Region
Template Framework} (RTF)~\cite{teodoro2014region}. Esse sistema pode analisar
imagens de tecidos de forma a extrair informações de elementos das imagens
(como células ou núcleos celulares) e algumas de suas características, como
tamanho, formato e aspectos da textura. Essas características são utilizadas
para desenvolver modelos morfológicos que podem ser utilizados para o
aprendizado de novos conhecimentos. Um exemplo da utilidade desta plataforma é
na análise de gliomas difusos, os núcleos das células do tumor são de interesse
significativo para a comunidade científica~\cite{gupta2005clarifying}.

Um fluxo de trabalho padrão desta plataforma consiste nas seguintes etapas: normalização, segmentação, computação de características, refinamento das características e classificação. Uma vez que um fluxo de trabalho busca analisar imagens de tecidos, temos que este sistema é dependente das imagens analisadas e consequentemente o resultado depende dos parâmetros de entrada do sistema. Para melhorar a qualidade das análises, portanto, é necessário estudar e compreender o impacto desses parâmetros no resultados. 

Para melhorarmos a qualidade das análises do RTF, iremos buscar estratégias que auxiliem na avaliação das características extraídas de uma imagem. Ela é uma etapa que influi diretamente na qualidade dos resultados das análises e ao ajustar os parâmetros buscando resultados com maior acurácia poderemos ter uma qualidade maior das análises. 

Atualmente, na etapa de segmentação, algumas estratégias já foram utilizadas para avaliar o impacto dos parâmetros de entrada em um fluxo de trabalho. As etapas de segmentação e de avaliação de características apesar de serem etapas distintas, compartilham atributos similares. Logo, podemos utilizar dos conhecimentos obtidos da utilização dessas estratégias na etapa de segmentação para direcionar um ponto de partida do estudo que será realizado na etapa de avaliação de características. Também temos que ambas etapas dependem da revisão subjetiva de especialistas e dado o tamanho das imagens, essa revisão torna-se bastante proibitiva. A partir das melhores estratégias estudadas e dos conhecimentos obtidos da etapa de segmentação, poderemos desenvolver modelos capazes de avaliar com alta precisão as características obtidas pelos métodos computacionais. Esses modelos podem ser utilizados na etapa de refinamento das características, pois seremos capazes de comparar os diversos resultados obtidos e assim extrapolar a necessidade da revisão de especialistas.

A etapa de refinamento de características pode ser separada em duas fases. A primeira, é a aplicação de métodos e técnicas de análise de sensibilidade (SA) e a segunda fase é o refinamento dos parâmetros. O objetivo final é encontrar dentro as possíveis combinações de resultados o conjunto de valores dos parâmetros que melhor influência no resultado final.

%A etapa de refinamento de características pode ser separada em duas fases. A primeira, é a aplicação de métodos e técnicas de análise de sensibilidade (SA). Esses métodos e técnicas permitem ao usuário entender e quantificar a variabilidade dos resultados de um sistema computacional e pode apontar quais são os parâmetros de entrada que causaram essas variabilidades. en

%Já a segunda fase é o refinamento dos parâmetros. Nesta fase os parâmetros de entrada tem seus valores alterados sistematicamente os resultados são comparados. Essas comparações são utilizadas para ajustar o conjunto de parâmetros.

\section{Justificativa}

%% problema
É imprescindível a busca de maneiras eficientes para se realizar processamento de alto desempenho para tarefas de análise de imagens. As análises de imagens de tecido de larga escala são problemas computacionalmente desafiadores. Um dos desafios é obter a escalabilidade dos vários fluxos de trabalho e/ou de grandes conjuntos de imagens com ganhos de qualidade. 

O \textit{National Institute of Health’s The Cancer Genome Atlas} (TCGA) disponibiliza uma grande quantidade de dados multimodais contendo imagens de patologia, radiologia, genômica e dados clínicos para o glioblastoma (gliomas cancerígenos) e outros tipo de tumores~\cite{mclendon2008comprehensive}. Atualmente, o RTF contém um banco de dados com mais de 600 imagens cada com uma com uma média de 400 mil núcleos por secção. O processo de caracterizar as morfologias celulares produz cerca de 1.5GB/seção de meta-dados que descrevem os parâmetros dos algoritmos, os limites dos objetos e as características das células~\cite{cooper2011morphological}.

%% modelo
Uma vez que a quantidade de dados gerada é elevada, não é humanamente possível avaliar todos resultados da caracterização das células e núcleos. Para isso, buscamos criar modelos capazes de avaliar o conjunto de características.

O desenvolvimento desses modelos é necessário para que as fases de refinamento de parâmetros e a análise de sensibilidade possam ser aplicados no contexto da computação de características. Após a implementação do modelo mais representativo capaz de avaliar a qualidade de um conjunto de características, podemos comparar os resultados obtidos entre os diversos conjuntos de parâmetros. 

%No contexto da análise de sensibilidade a utilização dos modelos é fundamentada na necessidade de quantificar a variabilidade dos resultados e verificar a influência dos parâmetros na aplicação. O refinamento de parâmetros é a fase com maior impacto nos recursos computacionais, uma vez que quanto mais combinações forem consideradas, melhor o espaço de busca é explorado. Porém, podemos ter parâmetros cuja influência nos resultados das análises seja baixa. Para verificar quais são os parâmetros mais com maior influência e eliminar os de menor influência pode-se utilizar a análise de sensibilidade. Assim, teremos um espaço de busca potencialmente menor e com o conjunto de parâmetros reduzidos diminui-se o impacto do refinamento de parâmetros nos recursos computacionais.

\section{Objetivos}

A partir do trabalho já realizado em~\cite{globalsensitibity}, pretende-se melhorar a otimização de avaliação de parâmetros com foco direcionado para a etapa de classificação das características. Dessa forma espera-se incrementar a qualidade das análises realizadas pelo sistema e, se possível, com ganhamos computacionais de tempo.


\subsection{Objetivos Específicos}

\begin{enumerate}
	\item  Gerar e implementar modelos de avaliação das características das células e núcleos;
\item  Atualizar os métodos e técnicas de análise de sensibilidade e refinamento de parâmetros disponíveis no RTF;
\item Otimizar a qualidade das análise das características das células e núcleo e consequentemente a do sistema como um todo.
\end{enumerate}


\section{Revisão da Literatura}


Na Seção~\ref{introducao} comentamos que buscamos impactar a qualidade das análises do RTF ao aprimorar a avaliação das características extraídas de uma imagem. Esse aprimoramento é dividido em duas fases, a análise de sensibilidade (SA) e o refinamento de parâmetros. Para que a aplicação dessas fases sejam viáveis em um sistema de análise de imagens é necessário propor e utilizar eficientemente SA e métodos de refinamento.


Já estão presentes, no RTF, os métodos de SA Variance-based Decomposition (VBD)~\cite{weirs2012sensitivity}~\cite{sobol2001global}  e o Morris One-At-A-Time design (MOAT)~\cite{morris1991factorial}~\cite{campolongo2007effective}. Para o refinamento de parâmetros, os métodos existentes são o Nelder-Mead simplex (NM), Parallel Rank Order (PRO), e o Algoritmo Genético de auto-refinamento. 

A maioria dos trabalhos recentes desenvolvidos para a otimização de parâmetros aplicam técnicas para modelos de segmentação específicos~\cite{kumar2003discriminative, szummer2008learning, mcintosh2007single, schultz2013open}. O sistema Tuner~\cite{torsney2011tuner} foca em algoritmos de segmentação em geral e cria modelos estatísticos que descrevem uma função de resposta da segmentação e utiliza de um modelo Gaussiano para explorar o espaço de busca. Já o sistema DAKOTA inclui método diversos de otimização de parâmetros, assim como métodos para modelagem surrogate, métodos de amostragem estatística de dados, estratégias de otimização baseada em surrogate e etc~\cite{giunta2002use}. Algumas outras opções que também serão analisadas como os métodos de otimização Particle swarm, de otimização by branch-and-fit, de otimização Bayesiana, algoritmos de Hit-and-Run.

Estes métodos serão utilizados como ponto de partida, pois nenhum deles é voltado para a análise de imagens médicas de patologia, logo não são diretamente aplicáveis. Além disso, não existe uma ferramenta ou estudo que aborda o problema de refinamento de parâmetros e/ou de análise de parâmetros voltado para entendimento de características das células de uma imagem, que é o foco desse trabalho.

Para o trabalho proposto as ferramentas já disponíveis no RTF e os métodos pré-citados serão incorporados para análise de sensibilidade e refinamento de parâmetros. Os resultados obtidos e as influências observadas serão analisados, buscando a maximização a qualidade dos resultados gerados pelo RTF. 

\section{Metodologia}
Inicialmente, será conduzida uma investigação mais detalhada da literatura sobre caracterização e previsão de custos de tarefas computacionais, sobre métodos de refinamento de parâmetros, com foco em algoritmos genéticos e os presentes na plataforma RTF, e sobre modelagem de características de células e núcleos celulares. Também será estudado mais profundamente o domínio de fluxos de trabalho de tratamento de imagens médicas. Em seguida, serão incorporados os modelos de características de células e núcleos celulares em conjunto com as técnicas de refinamento de parâmetros mais promissores da literatura e os já presentes na plataforma RTF.

\section{Plano de Trabalho}

Este trabalho será realizado em 5 etapas:
\begin{enumerate}
	\item Implementação da modelagem das características de células e núcleos celulares;
	\item Implementação de métodos de refinamento de parâmetros;
	\item Análise do reuso dos métodos de refinamento de parâmetros do RTF;
	\item Realização de testes para avaliar o desempenho obtido por cada otimização individualmente e em conjunto;
	\item Análise de resultados de testes e escrita da dissertação e artigos científicos.
\end{enumerate}


\section{Cronograma}
\begin{table}[!h]
	\centering
	\label{my-label}
	\begin{tabular}{|c|c|c|c|c|}
		\hline
		\multicolumn{1}{|l|}{Atividade} & \multicolumn{2}{l|}{2017} & \multicolumn{2}{l|}{2018} \\ \hline
		& 1\degree    & 2\degree    & 1\degree    & 2\degree    \\ \hline
		Cursar Disciplinas de Mestrado  & X           &  X           &             &             \\ \hline
		Revisão Bibliográfica           & X           & X           &             &             \\ \hline
		Implementação das tarefas X, Y  &             & X           & X           &             \\ \hline
		Implementação de W, Z     &             &             & X           &             \\ \hline
		Escrita Da Dissertação          &             &             & X           & X           \\ \hline
	\end{tabular}
\end{table}



%	\medskip

\newcommand{\BIBdecl}{\setlength{\itemsep}{0.2 em}}
\linespread{0.9}
	
\bibliographystyle{unsrt}%Used BibTeX style is unsrt
%\begin{scriptsize}
\bibliography{sample}
%\end{scriptsize}	
\end{document}
